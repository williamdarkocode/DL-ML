\documentclass[12pt, a4paper]{article}

\usepackage{listings}
\usepackage{color}
\usepackage{mathabx}



\title{NumPy Notes}
\author{William Darko}
\date{December 2020}

\pagenumbering{roman}

\begin{document}

\maketitle
\newpage

\tableofcontents

\newpage

\section{NumPy Basics}
\paragraph*{}
NumPy is stands for Numerical Python, and is a open source python Library containing multidimensional arrays and matrix data structures.

\subsection{Installing NumPy library}

If python is already installed, NumPy can be installed using:
\newline

\begin{lstlisting}
    conda install numpy
\end{lstlisting}

or

\begin{lstlisting}
    pip install numpy
\end{lstlisting}


\subsection{Importing NumPy}

Import NumPy into your python programme using:
\newline

\begin{lstlisting}
    import numpy as np
    
\end{lstlisting}


\subsection{Arrays}
Arrays are a central data-structure to the NumPy library. Some properties of arrays in the NumPy array are:

\begin{itemize}
    \item \textbf{dtype}: type of the elements in the array (given all elements in the array are of the same type, dtype will be the same value)
    \item \textbf{rank}: number of dimensions of the array
    \item \textbf{shape}: a tuple of non-negative integers providing the size of the array along each dimensions
    \item \textbf{ndarray}: is the NumPy n-dimensional array and used to represent both matrices and vectors
    \item \textbf{vector}: a 1-dimensional array
    \item \textbf{matrix}: a 2-dimensional array
    \item \textbf{tensor}: often a 3-dimensional array
    \item \textbf{axes}: in NumPy, dimensions are referred to as axes
\end{itemize}


\subsubsection{Creating Basic Arrays}
To create NumPy arrays, we can use one of the following:  
\newline

\begin{lstlisting}
    np.array(), np.zeros(), np.ones(), np.empty(), np.arange(), 
    np.linspace(), dtype
\end{lstlisting}

\begin{itemize}
    \item \textbf{np.array()}: Create a NumPy array by simply passing in a python list as an argument
    \begin{lstlisting}
        >>> import numpy as np
        >>> a = np.array([1, 2, 3])
    \end{lstlisting}

    \item \textbf{np.zeros()}: Create a NumPy array filled with zeros by passing in the length of the array as an argument
    \begin{lstlisting}
        >>> np.zeros(2)
        array([0., 0.])
    \end{lstlisting}

    \item \textbf{np.ones()}: Create a NumPy array filled with ones by passing in the length of the array as an argument
    \begin{lstlisting}
        >>> np.ones(3)
        array([1., 1., 1.])
    \end{lstlisting}

    \item \textbf{np.empty()}: Initialise a NumPy array populated with random values depending on the state of memory; the reason to use this over np.zeros, or np,ones, is speed.
    \begin{lstlisting}
        >>> np.empty(4)
        array([ 3.14, 42., 68 ])  # may vary
    \end{lstlisting}

    \item \textbf{np.arange(int n)}: Create NumPy array populated with integers from 0, to n-1
    \begin{lstlisting}
        >>> np.arange(4)
        array([0, 1, 2, 3])
    \end{lstlisting}

    \item \textbf{np.arange(first, last, step size)}: Create a NumPy array populated with numbers evenly spaced, starting at the first number, bounded by the last number
    \begin{lstlisting}
        >>> np.arange(2, 9, 2)
        array([2, 4, 6, 8])
    \end{lstlisting}

    \item \textbf{np.linspace(first, last, num)}: Create a NumPy array with linearly spaced numbers starting from the first number, ending at the last number, of length num
    \begin{lstlisting}
        >>> np.linspace(0, 10, num=5)
        array([ 0. ,  2.5,  5. ,  7.5, 10. ])
    \end{lstlisting}

    \item \textbf{dtype} keyword: The default data type for NumPy arrays is \textbf{np.float64} We can explicitly specify the data type we'd like to work with:
    \begin{lstlisting}
        >>> x = np.ones(2, dtype=np.int64)
        >>> x
        array([1, 1])
    \end{lstlisting}
\end{itemize}



\subsubsection{Manipulating and sorting arrays}
The functions of focus: \textbf{np.sort()}, and \textbf{np.concatenate()}
\begin{itemize}
    \item \textbf{np.sort()}: return a sorted copy of of same shape and type as an array by specifying the arguments:
    \subitem \textbf{a: \textit{array-like}} - the array to sort
    \subitem \textbf{axis: \textit{int or None, optional}} - The axis along which to sort; if \textit{None} is passed, the array is flattened (made 1 dimensional) before sorting. Default is -1, which sorts along last axis.
    \subitem \textbf{kind: \textit{(quicksort, mergesort, heapsort, stable) optional}} - choice of sorting algorithm to use. Default is \textit{quicksort}.
    \subitem \textbf{order: \textit{string or list of strings, optional}} - when the array is an array with fields defined, for instance:
    \begin{lstlisting}
    >>> dtype = [('name', 'S10'), ('height', float), ('age', int)]
    >>> values = [('Arthur', 1.8, 41), ('Lancelot', 1.9, 38),
    ('Galahad', 1.7, 38)]
    \end{lstlisting}
    the \textit{order} argument specifies which fields to compare first, or in other words the order in which to compare fields during sorting.
    \item \textbf{ndarray.sort()}: Sort an array in place; similar to \textbf{np.sort()} except this sorts an array in place and takes all the same arguments except for an array
    \item \textbf{np.concatenate()}: Where a and b are initialised arrays
    \begin{lstlisting}
        >>> np.concatenate(a,b)
    \end{lstlisting}
    combines the elements of both a, and b into a single array. More generally, np.concatenate() may take arguments of:
    \subitem \textbf{a1, a2, ...}: sequence of arrays to concatenate which must have the same shape, except in the dimension corresponding to axis (fist axis by default)
    \subitem \textbf{axis}: optional; a integer representing the axis along which the arrays will be concatenated. If \textit{None} is passed, default of 0th axis is used, and arrays are flattened before being concatenated
    \subitem \textbf{out}: optional; an ndarray as a destination to place the result. Must be of correct shape, matching what function will return.
    \subitem \textbf{dtype}: str or dtype; if provided the destination array will have this data type
    \subitem \textbf{casting}: one of the following (no, equiv, safe, same kind, unsafe)
\end{itemize}

\subsubsection{Knowing shape and size of array}
Using ndarray.ndim, ndarray.size, ndarray.shape

\begin{itemize}
    \item \textbf{ndarray.ndim}: tells you the number of axes or dimensions of the array
    \item \textbf{ndarray.size}: tells you the total number of elemets of the array. This is the product of the elements of the arrays shape
    \item \textbf{ndarray.shape}: displays a tuple of integers that indicate the number of elements along each axis, or dimension of the array.
    \begin{lstlisting}
        >>> array_example = np.array([[[0, 1, 2, 3],
        ...                            [4, 5, 6, 7]],
        ...
        ...                           [[0, 1, 2, 3],
        ...                            [4, 5, 6, 7]],
        ...
        ...                           [[0 ,1 ,2, 3],
        ...                            [4, 5, 6, 7]]])

        >>> array_example.ndim
        3
        >>> array_example.size
        24  // which is the product of 3 * 2 * 4
        >>> array_example.shape
        (3, 2, 4)

    \end{lstlisting}
\end{itemize}


\section{new section}



\end{document}
